%-------------------------------------------------------------------------------
%	SECTION TITLE
%-------------------------------------------------------------------------------
\cvsection{Presentations}


%-------------------------------------------------------------------------------
%	CONTENT
%-------------------------------------------------------------------------------
% \vspace{-0.3cm}
% \begin{small}\textit{* presenting author; \textsuperscript{+} mentored undergraduate}\end{small}

\cvsection{ }
%-------------------------------------------------------------------------------
\cvsubsection{Conférence invitée}
%-------------------------------------------------------------------------------

\begin{cvpubs}

%    \cvpub{mars 2024. L. Saussus, J. Gravier et B. Desachy, \textit{Mesurer l’enseignement des méthodes quantitatives en archéologie : essai d’état des lieux et retour d’expérience }. Invited Talk, Jounée Formations et enseignements des méthodes quantitatives en histoire organisée au sein de la revue Histoire & Mesures par Alessandro Stanziani, Paris, France.}

%    \cvpub{déc. 2023. J. Gravier et L. Hermenault, \textit{Articuler les apports de l’archéologie et de l’histoire grâce à ceux de la géographie. Retours d'expériences}. Invited Talk, Colloque "Sources ! De l'Archéologie aux fonds d'Archives", Agence Wallonne du Patrimoine, Archives de l'État de Belgique, Namur, Belgique.}

    \cvpub{juin 2023. \textit{“Events on linear network”, un même formalisme initial pour répondre à des questions diverses}. Invited Talk, Journée L’analyse des réseaux spatiaux en archéologie organisé par Constance Thirouard, GT 3.3 Territoires culturels et circulations des sociétés anciennes du Labex DynamiTe, Paris, France.}

    \cvpub{juin 2023. \textit{Dynamique des activités dans la ville durant un siècle : Paris au XIXe s}. Guest Lecture, séminaire systèmes complexes en sciences sociales organisé par Jean-Pierre Nadal, Henri Berestycki, Annick Vignes et Amandine Aftalion, EHESS, Paris, France (1h).}

    \cvpub{fév. 2023. \textit{Logiques de localisation des activités dans l’espace urbain. Études de cas et discussion à partir des données fines et en masse des annuaires commerciaux de Paris au XIXe siècle}. Guest Lecture, séminaire enjeux \& pratiques du numérique dans la recherche en histoire organisé par Bertrand Duménieu, Pascal Cristofoli, Raphaël Morera, Nicolas Veysset et Jean-Damien Généro, EHESS, Paris, France (2h).}

    \cvpub{nov. 2022. \textit{Des données fines et en masse de l’espace parisien au XIXe s. Analyse de l’évolution des activités urbaines au fil du siècle}. Guest Lecture, séminaire approches quantitatives et modélisation mathématiques en SHS organisé par Jean-Pierre Nadal, Annick Vignes et Julien Randon-Furling, PSL-Ecole des Chartes, Paris, France (2h).}

    \cvpub{nov. 2022. \textit{Dynamiques des villes en systèmes durant deux millénaires : approches croisées de la géographie et de l’archéologie}. Invited Talk, Journée d’étude Espaces et temporalités de la ville : approches croisées sur la construction des paysages urbains organisé par Barbara Chiti, EHESS UMR 8210 ANHIMA, UMR 7041 ArScAn, Paris, France.}

    \cvpub{avril 2022. \textit{Une ville en système avec d’autres : réseaux, interactions et dynamiques spatiales. Exemplification avec le cas de Noyon depuis le Ier s. apr. J.-C.}. Invited talk, Journée thématique de l’équipe TranSphères, UMR 7041 ArScAn, Nanterre, France.}

    \cvpub {nov. 2021. J. Charbonnier, Y. Kanhoush, J. Gravier, J. Giraud et al., \textit{Al-Ula Cultural Oasis Project: Results of a systematic survey in AlUla Valley (2019-2021)}. Invited talk, organisé par la Royal Commission for AlUla, Automn 2021 – Archaeological Projects Results, Al-Ula, Arabie Saoudite.}

    \cvpub {sept. 2021. J. Gravier, R. Housse, S. Boudia, J. Charbonnier, et Y. Kanhoush, \textit{Using LiDAR in Hyper-Arid Context: The Case of the Al-Ula Oasis in Saudi Arabia}. Invited talk, International Workshop: Archaeological LiDAR and ancient territories organisé par Antoine Dorison, Philippe Nondédéo, Grégory Pereira, Michelle Elliott et Christophe Petit, LabEx DynamiTe, Paris, France.}

    \cvpub {fév. 2020. \textit{Reconstituer les trajectoires démographiques de villes du Nord de la France du 1er au 21e siècle : enjeux et propositions méthodologiques}. Invited talk, Journée scientifique de l’UMR 8096 Archéologie des Amériques organisé par Marie-Charlotte Arnauld, Nanterre, France.}

    \cvpub{janv. 2020. \textit{Dynamiques socioéconomiques et disparités socio-spatiales de quatre métropoles mondiales (Londres, New-York, Paris, Tokyo) au XXIe s}. Guest Lecture, séminaire villes mondiales et métropoles : trajectoires locales dans un monde global organisé par Béatriz Fernandez, EHESS, Aubervilliers, France (2h).}

    \cvpub{nov. 2019. \textit{Analyser les interactions entre les villes sur 2000 ans : enjeux et propositions méthodologiques}. Guest Lecture, séminaire d’archéologie urbaine d’Élisabeth Lorans et de Xavier Rodier, Université de Tours, Tours, France (3h).}

    \cvpub{juin 2019. \textit{Géographie, archéologie et approches quantitatives : explorations à partir du cas de Noyon}. Guest Lecture, séminaire géographie et sciences sociales organisé par Nicolas Verdier et MarieVic Ozouf-Marignier, EHESS, Paris, France (2h).}

    \cvpub{mai 2019. \textit{Comparer et harmoniser des connaissances hétérogènes sur les relations entre les villes}. Guest Lecture, le 3e cycle 2018-2019 du séminaire SIG organisé par Éric Mermet, EHESS, Paris, France (2h).}
    
    \cvpub{fév. 2018. \textit{Appréhender la trajectoire d’une ville sur 2000 ans : le cas de Noyon en Picardie}. Guest Lecture, séminaire construction de l’espace médiéval dans l’ouest de l’Europe organisé par Brigitte Boissavit-Camus et Josiane Barbier, Université de Nanterre, Nanterre, France (1h30).}

    \cvpub{oct. 2016. J. Gravier et L. Hermenault \textit{Discussion autour de l’article : Faire dialoguer l’archéologie, la géographie et l’histoire, les enjeux de l’interdisciplinarité entre trois sciences humaines}. Guest Lecture, séminaire des doctorants du Laboratoire de Médiévistique Occidentale de Paris à l’Université Paris 1 – Panthéon-Sorbonne, Paris, France (1h30).}

    \cvpub{oct. 2014. \textit{Formalisation du temps des « objets urbains » pour l’étude diachronique des villes : une première esquisse}. Invited Talk, Atelier Ville : Fabrique urbaine, archéologie et modélisation organisé par Xavier Rodier, Réseau ISA – Information Spatiale et Archéologie, Laboratoire Archéologie et Territoires, UMR CITERES, MSH Val de Loire, MSH Dijon, Tours, France.}

    \cvpub{mars 2014. \textit{Systèmes de peuplement sur le temps long : une application aux villes du nord du Bassin parisien}. Guest Lecture, séminaire les jeudis de l’archéologie médiévale et moderne organisé par Bruno Desachy et Anne Nissen, Université Paris 1 – Panthéon-Sorbonne, Paris, France (1h).}
    
    \cvpub{fév. 2014. \textit{Système intra-urbain inscrit dans ses systèmes de villes. Le cas de Noyon dans le nord du Bassin parisien}. Guest Lecture, séminaire séminaire résilience de longue durée et réseaux informels de circulation organisé par Sandrine Robert, EHESS, Paris, France (1h30).}

    \cvpub{fév. 2013. \textit{Présentation d’une recherche en cours sur les systèmes et les réseaux urbains (Noyon, Beauvais, Saint-Quentin)}. Guest Lecture, séminaire les jeudis de l’archéologie médiévale et moderne organisé par Bruno Desachy et Anne Nissen, Université Paris 1 – Panthéon-Sorbonne, Paris, France (1h).}
    
    \cvpub{fév. 2012. \textit{Topographie historique de Noyon du Ier s. apr. J.-C. au début du XXIe s.}. Guest Lecture, séminaire les jeudis de l’archéologie médiévale et moderne organisé par Joëlle Burnouf, Université Paris 1 – Panthéon-Sorbonne, Paris, France (1h).}
\end{cvpubs}

\cvsection{ }
%-------------------------------------------------------------------------------
\cvsubsection{Colloque international (intervention sans actes)}
%-------------------------------------------------------------------------------

\begin{cvpubs}

    \cvpub{\textbf{J. Gravier}, « Time budgets and forms of preindustrial cities: time for empirical studies », présenté à 23rd European Colloquium on Theoretical and Quantitative Geography, Braga, Portugal, 17 septembre 2023.}

    \cvpub{J. Charbonnier, \textbf{J. Gravier}, A. Leschallier de Lisle, A. de Smet, S. Boudia, V. Koszarek, et Y. Kanhoush, « Exploring land-use in arid regions based on statistical analyses of the agricultural plots in Al-Ula oasis (Hejaz, KSA) », présenté à 29th EAA Annual Meeting, Queens University, Belfast, 31 août 2023.}

    \cvpub{J. Charbonnier et al., « Archaeological investigations in the Islamic oasis of Al-Ula (KSA): preliminary results », présenté à 13th International Congress on the Archaeology of the Ancient Near East (ICAANE), University of Copenhagen, 22 mai 2023.}

    \cvpub{\textbf{J. Gravier}, G. Gourret, Y. Kanhoush, et J. Charbonnier, « Expansion of an oasis from the end of the 19th century to the present day: what are the factors behind these dynamics? », présenté à UGI-IGU Paris 2022 Time for Geographers, Paris, France, 19 juillet 2022.}

    \cvpub{\textbf{J. Gravier}, « A city within its systems of cities over two thousand years », présenté à 21st European Colloquium on Theoretical and Quantitative Geography, Mondorf-Les-Bains, Luxembourg, 8 septembre 2019.}

    \cvpub{B. Desachy, C. Filet, A. Feugnet, C. Tomczyk, L. Hermenault, et \textbf{J. Gravier}, « Tenir compte des imprécisions chronologiques dans les décomptes d’évènements archéologiques par pas de temps », présenté à XVIIIe world UISPP Congress, Paris, France, 7 juin 2018.}

    \cvpub{\textbf{J. Gravier}, « An interdisciplinary approach of urban evolution over long periods of time: theoretical and methodological aspects », présenté à 19th European Colloquium in Theoretical and Quantitative Geography, Bari, Italie, 7 septembre 2015.}

\end{cvpubs}

\cvsection{ }
%-------------------------------------------------------------------------------
\cvsubsection{Colloque national (intervention sans actes)}
%-------------------------------------------------------------------------------

\begin{cvpubs}

    \cvpub{A. Bretagnolle, \textbf{J. Gravier}, C. Mimeur, et P. Poinsot, « Comment en est-on arrivé là ? L’évolution des services ferroviaires dans les lignes de desserte fine du territoire entre 2000 et 2015 », présenté à 3èmes Rencontres francophones Transport Mobilité, Marne-la-Vallée, France, 2 juin 2021.}

    \cvpub{R. Cura, \textbf{J. Gravier}, et L. Nahassia, « J’irai travailler sur vos tombes : quelles alternatives au terrain pour l’étude d’un espace qui n’existe plus ? », présenté à Journée des jeunes chercheurs de l’Institut de Géographie de Paris 2017 : géographie de l’alternatif, géographies alternatives ?, Paris, France, 5 avril 2017.}

    \cvpub{\textbf{J. Gravier}, « Interroger la taille des villes sur 2 000 ans. Population et surface des villes préindustrielles du nord de la France », présenté à 13e Rencontres ThéoQuant 2017, Nouvelles approches en Géographie Théorique & Quantitative, Besançon, France, 19 mai 2017.}

    \cvpub{\textbf{J. Gravier}, « Interroger la taille des villes sur le temps long : explorations statistiques et spatiales des surfaces et des populations de l’espace urbain dans le nord de la France », présenté à 5e Journées Informatique et Archéologie de Paris, Paris, France, 9 juin 2016.}

    \cvpub{R. Cura, \textbf{J. Gravier}, et L. Nahassia, « Modéliser des systèmes complexes entre géographie, archéologie et informatique : réflexions de trois jeunes chercheurs », présenté à XXIe journées de Rochebrune, Rencontres interdisciplinaires sur les systèmes complexes naturels et artificiels, Rochebrune, France, 23 janvier 2014.}
    
\end{cvpubs}

\cvsection{ }
%-------------------------------------------------------------------------------
\cvsubsection{Autres présentations}
%-------------------------------------------------------------------------------

\begin{cvpubs}

    \cvpub{juin 2023, \textit{Construction de catégories d’activités à partir des Annuaires de Paris au XIXe s. : éléments de discussion}, présenté à séminaire du LaDéHiS-CRH, EHESS, Paris, France.}

    \cvpub{mai 2023, \textit{Evolution of the urban structure of Paris during the 19th century. Building a temporal typology of urban activity categories}, présenté à 3e Journée de l’atelier SoDUCo-BnF, Paris, France, \href{https://hal.science/hal-04105159/}{hal-04105159}.}

    \cvpub{déc. 2022, \textit{Regard de... la géographie théorique et quantitative pour étudier les usages du concept de quartier en archéologie}, présenté à Faire ou ne pas faire (de) quartier ?, 5e journée d’étude d’Archéologie du fait urbain, Nanterre, France, \href{https://hal.science/hal-03977011/}{hal-03977011}.} 

    \cvpub{déc. 2022, \textit{SpacetimeLPP : un package R en cours de développement}, présenté à séminaire de parcours doctoral Systèmes d’informations et traitements des données archéologiques, Paris, France, \href{https://github.com/soduco/space_time_lpp}{package}.}

    \cvpub{nov. 2022, J. Calabuig Serra, J. Gravier, M. Guérois, T. Le Corre, T. Louail, H. Pecout, P. Pistre, O. Telle, C. Vacchiani-Marcuzzo, et J. Vallée, \textit{Des questions et des enjeux contemporains associés aux données de la recherche en SHS}, présenté à séminaire de l’équipe PARIS de l’UMR 8504 Géographie-cités, Aubervilliers, France, \href{https://hal.science/hal-03883797}{hal-03883797}.}

    \cvpub{nov. 2022, J. Gravier et M. Barthelemy, \textit{Exploitation des données SoDUCo : premiers éléments de l’évolution des activités parisiennes au fil du siècle}, présenté à 2e Journée de l’atelier SoDUCo-BnF, Paris, France, \href{https://hal.science/hal-03961814/}{hal-03961814}.}

    \cvpub{juin 2022, J. Gravier et G. Gourret, \textit{Évolution de l’oasis dans la seconde moitié du 20e siècle}, présenté à Séminaire interne UCOP-Archaïos, La chronologie de l’oasis d’Al-Ula, Paris, France.}

    \cvpub{mai 2022, \textit{Activités dans l’espace « infra » et dans le « temps moyen ». Localisations et évolutions à partir de calculs sur réseaux viaires}, présenté à séminaire de parcours doctoral Systèmes d’informations et traitements des données archéologiques, Paris, France.}

    \cvpub{sept. 2021, \textit{Reconstituer le réseau routier de l’oasis d’Al-Ula}, présenté à Journées d’études d’Archaïos, Paris, France.}

    \cvpub{nov. 2019, \textit{Récupérer automatiquement des coordonnées en R à partir de nom de lieux}, présenté à séminaire de parcours doctoral Systèmes d’informations et traitements des données archéologiques, Paris, France, \href{https://github.com/JGravier/sitRada/tree/master/Recuperation_coord_X_Y}{notebook}.}
    
    \cvpub{avril 2019, \textit{Le package sf, R-econfiguration de l’information spatiale}, présenté à séminaire de parcours doctoral Systèmes d’informations et traitements des données archéologiques, Paris, France.}

    \cvpub{janv. 2019, \textit{NERVilles : une BDD pour étudier les relations entre des villes sur le temps long}, présenté à séminaire de parcours doctoral Systèmes d’informations et traitements des données archéologiques, Paris, France.}

    \cvpub{mars 2018, \textit{Retour aux sources}, présenté à séminaire de parcours doctoral Systèmes d’informations et traitements des données archéologiques, Paris, France, \href{https://sitrada.hypotheses.org/184}{présentation et compte-rendu}.}

    \cvpub{janv. 2018, \textit{Interactions potentielles entre les lieux : application du modèle gravitaire aux villes du nord de la France}, présenté à séminaire de parcours doctoral Systèmes d’informations et traitements des données archéologiques, Paris, France, \href{https://rpubs.com/JGravier/347806}{notebook}.}

    \cvpub{juin 2017, \textit{Exploration spatio-temporelle de la ville de Noyon dans des systèmes de villes sur 2000 ans. Pour l’étude de la ville sur le temps long par le croisement de l’archéologie et de la géographie}, présenté à séminaire de l’équipe PARIS de l’UMR 8504 Géographie-cités, Paris, France.}

    \cvpub{avril 2017, \textit{Proposition méthodologique de traitement de données typo-chronologiques : analyse multidimensionnelle et visualisation de classes}, présenté à séminaire de parcours doctoral Systèmes d’informations et traitements des données archéologiques, Paris, France, \href{https://github.com/JGravier/sitRada/tree/master/VisuAnalyseMultidimensionnelle-Classes}{analyse et compte-rendu}.}

    \cvpub{janv. 2017, \textit{Tableaux typo-chronologiques et heatmaps avec R : visualiser et gérer les effets de sources}, présenté à séminaire de parcours doctoral Systèmes d’informations et traitements des données archéologiques, Paris, France, \href{https://rpubs.com/JGravier/254083}{notebook}.}

    \cvpub{mai 2016, \textit{Un peu d’analyse spatiale avec R}, présenté à séminaire de parcours doctoral Systèmes d’informations et traitements des données archéologiques, Paris, France, \href{https://hal.science/hal-02456921}{hal-02456921}.}

    \cvpub{mai 2016, \textit{AnalyseSHS du PIREH (Pôle Informatique de Recherches et d’Enseignement en Histoire de Paris 1)}, présenté à séminaire de parcours doctoral Systèmes d’informations et traitements des données archéologiques, Paris, France, \href{https://hal.science/hal-02456921}{hal-02456921}.}

    \cvpub{avril 2016, \textit{Interroger la taille des villes sur le temps long : premiers éléments de discussions}, présenté à séminaire de parcours doctoral Systèmes d’informations et traitements des données archéologiques, Paris, France, \href{https://hal.science/hal-02456921}{hal-02456921}.}

    \cvpub{juin 2016, J. Gravier, L. Nahassia, et L. Sanders, \textit{Système de peuplement}, présenté à journées d’étude du groupe de travail systèmes de peuplement sur le temps long, LabEx DynamiTe, Paris, France.}

    \cvpub{janv. 2016, \textit{Le package Explor sur R}, présenté à séminaire de parcours doctoral Systèmes d’informations et traitements des données archéologiques, Paris, France, \href{https://hal.science/hal-02456921}{hal-02456921}.}

    \cvpub{janv. 2016, \textit{Construction de découpages spatio-temporels : proposition d’application à un cas d’espace intra-urbain}, présenté à journées d’étude du groupe de travail systèmes de peuplement sur le temps long, LabEx DynamiTe, Dourdan, France.}

    \cvpub{oct. 2014, \textit{Exploration de la transférabilité du concept de transition TransMonDyn à l’étude de la ville de Noyon sur le temps long}, présenté à journées d’étude retour sur la conceptualisation et la formalisation des transitions du projet ANR TransMonDyn, Tours, France.}
    
    \cvpub{avril 2014, \textit{Un système intra-urbain inscrit dans ses systèmes de villes sur le temps long : Noyon dans ses réseaux de villes du nord du Bassin parisien, de l’Antiquité à nos jours}, présenté à séminaire de l’équipe PARIS de l’UMR 8504 Géographie-cités, Paris, France.}
    
    \cvpub{mars 2014, \textit{Traitement des temporalités de l’unité stratigraphique à l’entité urbaine : avancement de travaux}, présenté à séminaire de parcours doctoral Systèmes d’informations et traitements des données archéologiques, Paris, France.}

    \cvpub{janv. 2014, R. Cura, J. Gravier, et L. Nahassia, \textit{Modéliser des transitions : réflexions croisées de trois jeunes chercheurs en géographie, archéologie et informatique}, présenté à journées d’étude retour sur notre démarche et nos avancées : réflexivité et regards extérieurs du projet ANR TransMonDyn, Lyon, France.}

    \cvpub{janv. 2014, \textit{Système intra-urbain inscrit dans ses systèmes de villes. Approche diachronique et multiscalaire}, présenté à journées d’étude du groupe de travail systèmes de peuplement sur le temps long, LabEx DynamiTe, Paris, France.}

    \cvpub{juin 2013, \textit{Un système intra-urbain inscrit dans ses systèmes de villes : le cas de la ville de Noyon}, présenté à journées d’étude comparaison des transitions, transférabilité, généricité du projet ANR TransMonDyn, Sommières, France.}

    \cvpub{mars 2013, \textit{La ville de Noyon : système urbain dans un système de villes. Approche diachronique et multiscalaire}, présenté à séminaire de parcours doctoral Systèmes d’informations et traitements des données archéologiques, Paris, France.}


\end{cvpubs}
