%-------------------------------------------------------------------------------
%	SECTION TITLE
%-------------------------------------------------------------------------------
\cvsection{Publications}


%-------------------------------------------------------------------------------
%	CONTENT
%-------------------------------------------------------------------------------


\cvsection{ }
%---------------------------------------------------------
\cvsubsection{Article dans revue à comité}
%---------------------------------------------------------

\begin{cvpubs}
    \cvpub{C. Plumejeaud-Perreau, L. Nahassia, et \textbf{J. Gravier}, « Appréhender le changement des catégories pour l’étude d’une dynamique spatiale sur le temps long », \textit{Revue Internationale de Géomatique}, vol. 31, nᵒ 1/2, janv. 2022, doi: \href{https://doi.org/10.3166/rig31.47-80}{10.3166/rig31.47-80}.}
    
    \cvpub{C. Plumejeaud-Perreau, M. Fargette, \textbf{J. Gravier}, T. Libourel, E. Masson, H. Mathian, L. Nahassia, L. Nuninger, X. Rodier, L. Sanders, et E. Saux, « Introduction », \textit{Revue Internationale de Géomatique}, vol. 31, nᵒ 1/2, janv. 2022, doi: \href{https://doi.org/10.3166/rig31.7-19}{10.3166/rig31.7-19}.}

    \cvpub{É. Cavanna, B. Desachy, C. Gorin, \textbf{J. Gravier}, L. Hermenault, et Collectif Afu, « Le projet collectif Archéologie du fait urbain (Afu) : un nouvel espace de discussion », \textit{Les Nouvelles de l’archéologie}, nᵒ 164, 2021, doi: \href{https://doi.org/10.4000/nda.12209}{10.4000/nda.12209}.}

    \cvpub{\textbf{J. Gravier}, « Représenter et analyser les positions relatives des villes sur 2 000 ans », \textit{Mappemonde}, vol. 129, 2020, doi: \href{https://doi.org/10.4000/mappemonde.4562}{10.4000/mappemonde.4562}. Article faisant suite à la présélection de ma thèse par la section 23 du CNU au prix Mappemonde, n’ayant pas fait l’objet d’une évaluation par le comité de lecture de la revue.}

    \cvpub{H. Dulauroy-Lynch et \textbf{J. Gravier}, « La “synthèse archéologique urbaine”, l’exemple de Noyon. Un travail de recherche au service des archéologues et des habitants », \textit{Les Nouvelles de l’archéologie}, nᵒ 136, 2014, doi: \href{https://doi.org/10.4000/nda.12209}{10.4000/nda.2508}.}
\end{cvpubs}

\cvsection{ }
%---------------------------------------------------------
\cvsubsection{Édition de numéro spécial dans revue à comité}
%---------------------------------------------------------

\begin{cvpubs}
    \cvpub{C. Plumejeaud-Perreau, M. Fargette, \textbf{J. Gravier}, T. Libourel, E. Masson, H. Mathian, L. Nahassia, L. Nuninger, X. Rodier, L. Sanders, et E. Saux, Éd., Ontologies et dynamiques spatiales. Complémentarité des approches symboliques et numériques. \textit{Revue Internationale de Géomatique}, vol. 31. Paris: Lavoisier, 2022.}

    \cvpub{M. E. Castiello, D. Mouralis, \textbf{J. Gravier}, et L. Nahassia, Éd., Special Collection: Spatial Computation in Archaeology and History, \textit{Journal of Computer Application in Archaeology}. 2021.}

\end{cvpubs}

\cvsection{ }
%---------------------------------------------------------
\cvsubsection{Chapitre d’ouvrage chez éditeur à comité}
%---------------------------------------------------------

\begin{cvpubs}
    \cvpub{\textbf{J. Gravier}, L. Nahassia, D. Michelet, N. Verdier, et M. Olteanu, « Trajectoire », in \textit{Le temps long du peuplement. Concepts et mots clés}, L. Sanders, A. Bretagnolle, P. Brun, M.-V. Ozouf-Marignier, et N. Verdier, Éd., Tours: Presses universitaires François Rabelais, 2020, p. 107 à 125.}
    
    \cvpub{L. Nahassia, \textbf{J. Gravier}, D. Michelet, N. Verdier, et M. Olteanu, « Processus », in \textit{Le temps long du peuplement. Concepts et mots clés}, L. Sanders, A. Bretagnolle, P. Brun, M.-V. Ozouf-Marignier, et N. Verdier, Éd., Tours: Presses universitaires François Rabelais, 2020, p. 89 à 106.}

    \cvpub{N. Verdier, \textbf{J. Gravier}, L. Nahassia, et D. Michelet, « Échelles : niveaux d’observation et de représentation », in \textit{Le temps long du peuplement. Concepts et mots clés}, L. Sanders, A. Bretagnolle, P. Brun, M.-V. Ozouf-Marignier, et N. Verdier, Éd., Tours: Presses universitaires François Rabelais, 2020, p. 67 à 88.}

    \cvpub{D. Michelet, \textbf{J. Gravier}, L. Nahassia, et N. Verdier, « Temps, espaces et mots connexes », in \textit{Le temps long du peuplement. Concepts et mots clés}, L. Sanders, A. Bretagnolle, P. Brun, M.-V. Ozouf-Marignier, et N. Verdier, Éd., Tours: Presses universitaires François Rabelais, 2020, p. 47 à 66.}

    \cvpub{S. Rey-Coyrehourcq, R. Cura, L. Nuninger, \textbf{J. Gravier}, L. Nahassia, et R. Hachi, « Vers une recherche reproductible dans un cadre interdisciplinaire : enjeux et propositions pour le transfert du cadre conceptuel et la réplication des modèles », in \textit{Peupler la Terre}, L. Sanders, Éd., Tours: Presses universitaires François-Rabelais, 2017, p. 409 à 434. doi: \href{https://doi.org/10.4000/books.pufr.10647}{10.4000/books.pufr.10647}.
    \textbf{Traduction:} S. Rey-Coyrehourcq, R. Cura, L. Nuninger, J. Gravier, L. Nahassia, et R. Hachi, « Chapter 15: Towards reproducible research in an interdisciplinary framework: issues and propositions regarding the transfer of the conceptual framework and the replication of models », in \textit{Settling the World : From Prehistory to the Metropolis Era}, L. Sanders, Éd., Tours: Presses universitaires François-Rabelais, 2021. [En ligne]. Disponible sur: \href{http://books.openedition.org/pufr/20002}{http://books.openedition.org/pufr/20002}}

    \cvpub{L. Nuninger et al., « Un cadre conceptuel générique pour décrire des transitions dans les systèmes de peuplement. Application à un corpus de douze transitions entre 70000 BP et 2050 », in \textit{Peupler la Terre}, L. Sanders, Éd., Tours: Presses universitaires François-Rabelais, 2017, p. 55 à 88. doi: \href{https://doi.org/10.4000/books.pufr.10527}{10.4000/books.pufr.10527}.
    \textbf{Traduction:} L. Nuninger et al., « Chapter 2: A generic conceptual framework for describing transitions in settlement systems : Application to a corpus of twelve transitions between 70 000 BP and 2050 », in \textit{Settling the World : From Prehistory to the Metropolis Era}, L. Sanders, Éd., Tours: Presses universitaires François-Rabelais, 2021. [En ligne]. Disponible sur: \href{https://books.openedition.org/pufr/19740}{https://books.openedition.org/pufr/19740}}

    \cvpub{Q. Borderie, B. Desachy, J. Delahaye, \textbf{J. Gravier}, et A. Pinhède, « Synthèses Archéologiques Urbaines : un projet en cours », in \textit{Archéologie de l’espace urbain}, É. Lorans et X. Rodier, Éd., Tours / Paris: Presses universitaires François-Rabelais / Cths, 2014, p. 262 à 276. doi: \href{https://doi.org/10.4000/books.pufr.7678}{10.4000/books.pufr.7678}}

\end{cvpubs}

\cvsection{ }
%---------------------------------------------------------
\cvsubsection{Actes de colloque ayant fait l’objet d’une évaluation par un comité}
%---------------------------------------------------------

\begin{cvpubs}
    \cvpub{J. Charbonnier, Y. Kanhoush, \textbf{J. Gravier}, G. Gourret, et al., « Mapping an Arabian oasis: first results of the UCOP systematic survey of al\text{-}ʿUlā (AlUla) Valley (2019–2021) », in Revealing Cultural Landscapes in North\text{-}West Arabia, R. Foote, M. Guagnin, I. Périssé, et S. Karacic, Éd., in \textit{Supplement to Volume 51 of the Proceedings of the Seminar for Arabian Studies}, Oxford: Archaeopress, 2022, p. 51 à 81.} 

    \cvpub{\textbf{J. Gravier}, « Reconfigurations territoriales des services publics dans les villes françaises (2009-2018) », in \textit{CIST 2020 - Population, temps, territoires/Population, Time, Territories}, Aubervilliers: Collège international des sciences territoriales (CIST), 2020, p. 263 à 268. [En ligne]. Disponible sur: \href{https://hal.archives-ouvertes.fr/hal-03114137}{hal-03114137}.}

    \cvpub{\textbf{J. Gravier} et L. Hermenault, « Faire dialoguer l’archéologie, la géographie et l’histoire : les enjeux de l’interdisciplinarité entre trois sciences sociales », \textit{Archéo.doct : L’archéologie, science plurielle}, vol. 11, 2018, doi: \href{https://doi.org/10.4000/books.psorbonne.7087}{10.4000/books.psorbonne.7087}.}

    \cvpub{\textbf{J. Gravier}, « Délimiter un espace d’étude sur 2 000 ans : complémentarité des approches territoriale et réticulaire », in \textit{CIST 2018 - Représenter les territoires/Representing territories}, Rouen: Collège international des sciences territoriales (CIST), 2018, p. 389 à 395. [En ligne]. Disponible sur: \href{https://hal.archives-ouvertes.fr/hal-01854367}{hal-01854367}.}

    \cvpub{\textbf{J. Gravier}, « Recognizing Temporalities in Urban Units from a Functional Approach: Three Case Studies », in \textit{CAA 2014, 21st Century Archaeology, Concepts, Methods and Tools. Proceedings of the 42nd Annual Conference on Computer Applications and Quantitative Methods in Archaeology}, F. Giligny, F. Djindjian, L. Costa, P. Moscati, et S. Robert, Éd., Oxford: Archaeopress, 2015, p. 371 à 380. [En ligne]. Disponible sur: \href{https://hal.archives-ouvertes.fr/hal-01474197}{hal-01474197}.}

\end{cvpubs}

\cvsection{ }
% %---------------------------------------------------------
\cvsubsection{Autres publications}
% %---------------------------------------------------------

Encadré dans le livre \textit{Le temps long du peuplement : concepts et mots-clés}, Presses universitaires François Rabelais, Tours, 2020 :

\begin{cvpubs}
    \cvpub{\textbf{J. Gravier} et L. Nahassia, « Dossier 2.B. La distance-temps comme échelle de la croissance urbaine », p. 84 à 88}

    \cvpub{L. Nahassia et \textbf{J. Gravier}, « Dossier 3.B. Cycle, théorie de la destruction créatrice de Schumpeter et systèmes de villes », p. 104 à 106}

    \cvpub{L. Nahassia et \textbf{J. Gravier}, « Dossier 4.A. Conceptualisation de trajectoires à partir de données GPS par Élodie Buard », p. 122 à 123}

    \cvpub{\textbf{J. Gravier} « Dossier 4.B. Trajectoire de la ville de Noyon du Ier siècle de notre ère jusqu’à aujourd’hui », p. 123 à 125}
\end{cvpubs}


Poster :

\begin{cvpubs}

    \cvpub{A. Defauconpret, \textbf{J. Gravier}, Y. Kanhoush, et J. Charbonnier, « Geolocating old photographs from Al-Ula oasis: spatial analysis in the framework of the Al-Ula Cultural Oasis Project », présenté à \textit{55th Seminar for Arabian studies}, Humboldt University, Berlin, août 2022.}

    \cvpub{\textbf{J. Gravier}, « Des changements fonctionnels de l’espace urbain. Le cas de Noyon, Ier s. apr. J.-C.-XXIe s. », présenté à Rencontres du GdR MoDyS, figshare, Frasne, France, novembre 2013. doi: \href{https://doi.org/10.6084/m9.figshare.1449180.v3}{10.6084/m9.figshare.1449180.v3}.}
    
\end{cvpubs}

\cvsection{ }
%---------------------------------------------------------
\cvsubsection{Jeux de données (sélection)}
%---------------------------------------------------------

\begin{cvpubs}

    \cvpub{\textbf{J. Gravier}, « Dataset and data analysis of “Scaling and dynamics of activities in a growing city” », Zenodo, 2023. doi: \href{https://doi.org/10.5281/zenodo.8388102}{10.5281/zenodo.8388102}.}
    
    \cvpub{\textbf{J. Gravier}, « Facilities of Public Services in Metropolitan France (2009, 2013, 2018) ». Zenodo, 2023. doi: \href{https://doi.org/10.5281/zenodo.8163543}{10.5281/zenodo.8163543}.}

    \cvpub{GeoHistoricalData, « Annuaires historiques parisiens, 1798-1914. Extraction structurée et géolocalisée à l’adresse des listes nominatives par ordre alphabétique et par activité dans les volumes numérisés ». NAKALA, 2023. doi: \href{https://doi.org/10.34847/nkl.98eem49t}{10.34847/nkl.98eem49t}.}

    \cvpub{P. Cristofoli et \textbf{J. Gravier}, « Populations des quartiers de Paris (1801-1911) ». NAKALA, 2023. doi: \href{https://doi.org/10.34847/nkl.e173c93p}{10.34847/nkl.e173c93p}.}

    \cvpub{\textbf{J. Gravier}, « Dataset of infra-urban accessibility of northern French cities in the early 19th century ». NAKALA, 2023. doi: \href{https://doi.org/10.34847/nkl.ff1d91yw}{10.34847/nkl.ff1d91yw}.}
    
    \cvpub{\textbf{J. Gravier}, « Districts of Paris (1860-1919) ». NAKALA, 2022. doi: \href{https://doi.org/10.34847/nkl.a57506s3}{10.34847/nkl.a57506s3}.}

    \cvpub{\textbf{J. Gravier}, « Areas and populations of urban settlements ». Figshare, Paris, 2018. doi: \href{https://doi.org/10.6084/m9.figshare.5632015.v3}{10.6084/m9.figshare.5632015.v3}.}

    \cvpub{\textbf{J. Gravier}, « Surface et population des villes en système avec Noyon sur le temps long ». Figshare, 2018. doi: \href{https://doi.org/10.6084/m9.figshare.7172333.v1}{10.6084/m9.figshare.7172333.v1}.}
    
    \cvpub{J. Perret et al., « The 18th century Cassini roads and cities dataset ». Harvard Dataverse, 2015. doi: \href{https://doi.org/10.7910/DVN/28674}{10.7910/DVN/28674}.}

    \cvpub{\textbf{J. Gravier}, « Table poléométrique - complete data set ». Figshare, 15 juin 2015. doi: \href{https://doi.org/10.6084/m9.figshare.1449225.v2}{10.6084/m9.figshare.1449225.v2}.}

\end{cvpubs}

\cvsection{ }
% %---------------------------------------------------------
\cvsubsection{Rapports récents} % perso: faire une selection
% %---------------------------------------------------------

\begin{cvpubs}

    \cvpub{\textbf{J. Gravier}, « Restructuring of the General Field Database », in \textit{UCOP: Al-Ula Cultural Oasis Project. Periodic Report}, J. Charbonnier, Y. Kanhoush, et J. Giraud, Éd., Paris, 2022, p. 568 à 586.}
    
    \cvpub{G. Gourret et \textbf{J. Gravier}, « Improving the ontology of UCOP field GIS », in \textit{UCOP: Al-Ula Cultural Oasis Project. Periodic Report}, J. Charbonnier, Y. Kanhoush, et J. Giraud, Éd., Paris, 2022, p. 587 à 593.}
    
    \cvpub{\textbf{J. Gravier}, A. Defauconpret, G. Gourret, A. de Smet, M. Esso, et R. Housse, « New resources to study the spatial dynamics of the oasis of Al-Ula », in \textit{UCOP: Al-Ula Cultural Oasis Project. Periodic Report}, J. Charbonnier, Y. Kanhoush, et J. Giraud, Éd., Paris, 2022, p. 594 à 627.}

    \cvpub{\textbf{J. Gravier} et C. Thirouard, « Les implantations des sites archéologiques et ethnohistoriques par rapport aux éléments hydrographiques du nord-ouest de l’Alaska », in \textit{Spatialité de l’Habitat et Régionalisation sur le Kobuk (SHARK) : la culture de Thulé du Cap Espenberg à la séquence Woodland (1100 – 1850 apr. J.-C.). Rapport périodique 2022}, C. Alix, C. Thirouard, J. Gravier, O. K. Mason, et J. Taïeb, Éd., Paris, 2022, p. 18 à 36.}

    \cvpub{\textbf{J. Gravier}, G. Gourret, S. Boudia, et V. Bernollin, « Archaeological Survey and Data Management », in \textit{UCOP: Al-Ula Cultural Oasis Project. Periodic Report}, J. Charbonnier, Y. Kanhoush, J. Goy, et J. Giraud, Éd., Paris, 2021, p. 517 à 555.}

    \cvpub{C. Alix, C. Thirouard, \textbf{J. Gravier}, O. K. Mason, et J. Taïeb, \textit{Spatialité de l’Habitat et Régionalisation sur le Kobuk (SHARK) : la culture de Thulé du Cap Espenberg à la séquence Woodland (1100 – 1850 apr. J.-C.). Rapport périodique 2021.} Paris, 2021.}

    \cvpub{\textbf{J. Gravier} et B. Fernandez (collab.), \textit{Panorama socio-économique des métropoles de Paris, Tokyo, Londres, New York et Washington au 21e s.}, Fondation France-Japon – EHESS, Toyota Motor Corporation, Paris, 2019.}
    
\end{cvpubs}


\cvsection{ }
% %---------------------------------------------------------
\cvsubsection{Publication acceptée}
% %---------------------------------------------------------

\begin{cvpubs}
    \cvpub{\textit{À paraître}, J. Charbonnier, Y. Kanhoush, E. Devaux, \textbf{J. Gravier}, V. Bernollin, et T. Hofstetter, « AlUla Old Town and Oasis – Oasis Farms: preliminary study and first results from the AlUla Cultural Oasis Project (Kingdom of Saudi Arabia) », in \textit{13th World Congress on Earthen Architectural Heritage}, Santa Fe, USA: Getty Conservation Institute, National Park Service, Vanishing Treasures Program and University of Pennsylvania, Stuart Weitzman School of Design, 2022.}

%    \cvpub{\textbf{Accepté, en cours de reviewing}, \textbf{J. Gravier}, « Réinterroger les relations entre forme urbaine, accessibilité intra-urbaine et taille démographique des villes préindustrielles par la mesure », \textit{Cybergeo: European Journal of Geography}.}
\end{cvpubs}

