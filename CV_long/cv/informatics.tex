%-------------------------------------------------------------------------------
%	SECTION TITLE
%-------------------------------------------------------------------------------
\cvsection{Informatique et données}

%-------------------------------------------------------------------------------
%	CONTENT
%-------------------------------------------------------------------------------
\cvsubsection{Planification, création et collecte}

\begin{small} \color{black}
\textit{Méthode :} DMP, enquête par questionnaire, bases de données relationnelles spatiales et non spatiales, ontologie, web scrapping (API et page HTML fixe) \\
\textit{Outil :} Opidor, Framaforms, UML, SQL, PostgreSQL et PostGIS (via pgAdmin), QField, module Base de LibreOffice, MS Access, Protégé, GraphDB, Silk-LSL, R (en particulier rverst)
\end{small}

\cvsubsection{Organisation, traitement et analyse}

\begin{small} \color{black}
\textit{Méthode :}  nettoyage et enrichissement de données, statistique, géomatique, réseau et graphe, text mining, parallélisation, visualisation statique et dynamique, traitement graphique et cartographique, écriture \\
\textit{Outil :} R -- en particulier la famille du tidyverse, tidygeocoder, stats, datatable, ade4, sf, famille de spatstat, lidR, terra, igraph, tm, tidytext, parallel et doParallel (dont utilisation de la ferme de calcul du CC-IN2P3 grâce à Huma-Num), shiny, tmap, leaflet --, QGIS, GRASS, SAGA, Inkscape, LaTeX (Overleaf et TeXstudio)
\end{small}

\cvsubsection{Conservation, accès et réutilisation}

\begin{small} \color{black}
\textit{Méthode :} documentation et archivage de données, versionnement et collaboration, création de site internet, reproductibilité et programmation lettrée, développement de package R \\
\textit{Outil :} Zenodo, Nakala, Harvard Dataverse, Git (principalement GitHub, GitLab), WordPress, Quarto Websites, Quarto, Markdown (R Markdown, HackMD), R (roxygen2)
\end{small}

